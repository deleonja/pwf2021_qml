\section{Exercise sheet 3}
\subsection{Exercise 1 (Hadamard test)}
Mathematically, what is happening in the circuit is
\begin{subequations}
\begin{align}
\qty(H\otimes\mathbbm{1})\qty(\dyad{0}{0}\otimes\mathbbm{1}+\dyad{1}{1}\otimes U)\qty(H\otimes \mathbbm{1})
\Big[\ket{0}\otimes \ket{\psi}\Big]&=
\qty(H\otimes\mathbbm{1})\qty(\dyad{0}{0}\otimes\mathbbm{1}+\dyad{1}{1}\otimes U)
\Big[\frac{1}{\sqrt{2}}\qty(\ket{0}+\ket{1})\ket{\psi}\Big]\\
&=\qty(H\otimes\mathbbm{1})\qty[\frac{1}{\sqrt{2}}\qty(\ket{0}\ket{\psi}+\ket{1}U\ket{\psi})]\\
&=\frac{1}{\sqrt{2}}
\qty(\ket{0}\qty(\mathbbm{1}+U)\ket{\psi}+
	\ket{1}\qty(\mathbbm{1}-U)\ket{\psi})
\end{align}
\end{subequations}
To find the probability the qubit $0$ in $\ket{0}$ is
\begin{align}
\matrixel{\psi'}{\qty(\dyad{0}{0}\otimes\mathbbm{1})}{\psi'}=
	\frac{1}{4}\qty(2+\matrixel{\psi}{U}{\psi}+\matrixel{\psi}{U^{\dagger}}{\psi})
\end{align}
Now, noting that $\qty(\matrixel{\psi}{U}{\psi})^*=\matrixel{\psi}{U^{\dagger}}{\psi}$,
\begin{align}
p(0)=\matrixel{\psi'}{\qty(\dyad{0}{0}\otimes\mathbbm{1})}{\psi'}=
	\frac{1}{2}\bigg(1+\Re \Big[\matrixel{\psi}{U}{\psi}\Big]\bigg).
\end{align}

\subsection{Exercise 2 (Error analysis)}

\subsection{Exercise 3 (Useful equality)}\noindent
The right-hand of the circuit equation in fig. 1(b) writes
\begin{align}
CNOT(0,1)\cdot CNOT(1,0)\cdot CNOT(0,1),
\end{align}
where $CNOT(c,t)$ is notation for the $CNOT$ gate with control qubit $c$
and target qubit $t$. The action of the $CNOT$ gate over the computational 
basis is $\ket{c}\ket{t}\mapsto\ket{c}\ket{t\oplus c}$ (with $\oplus$ sum
mod 2). Furthermore,
\begin{align}
CNOT(0,1)&=
\left(
\begin{array}{cccc}
 1 & 0 & 0 & 0 \\
 0 & 1 & 0 & 0 \\
 0 & 0 & 0 & 1 \\
 0 & 0 & 1 & 0 \\
\end{array}
\right),&
CNOT(1,0&)=
\left(
\begin{array}{cccc}
 1 & 0 & 0 & 0 \\
 0 & 0 & 0 & 1 \\
 0 & 0 & 1 & 0 \\
 0 & 1 & 0 & 0 \\
\end{array}
\right).
\end{align}
This exercised was done in Mathematica with the following code
\begin{verbatim}
(*Define CNOT(0,1), CNOT(1,0) and Swap*)
MatrixForm[
 CNOT0 = KroneckerProduct[M0, IdentityMatrix[2]] + 
   KroneckerProduct[M1, Pauli[1]]]
MatrixForm[
 CNOT1 = KroneckerProduct[IdentityMatrix[2], M0] + 
   KroneckerProduct[Pauli[1], M1]]
MatrixForm[
 Swap = {{1, 0, 0, 0}, {0, 0, 1, 0}, {0, 1, 0, 0}, {0, 0, 0, 1}}]
(*Do matrix multiplication*)
MatrixForm[CNOT0.CNOT1.CNOT0]
CNOT0.CNOT1.CNOT0 == Swap
Out:= True
\end{verbatim}

\subsection{Exercise 4 (Useful equality)}
$SWAP$ gate may be written in Dirac notation as
\begin{align}\label{eq:SWAP}
SWAP=\dyad{00}{00}+\dyad{01}{10}+\dyad{10}{01}+\dyad{11}{11}.
\end{align}
It the follows,
\begin{subequations}
\begin{align}
SWAP\Big[ \ket{\psi	}\otimes \ket{\phi}\Big]&=
	SWAP\Big[ c_0b_0\ket{00}+c_0b_1\ket{01}+c_1b_0\ket{10}+c_1b_1\ket{11} \Big]\\
&=c_0b_0\ket{00}+c_0b_1\ket{10}+c_1b_0\ket{01}+c_1b_1\ket{11}\\
&=\big( b_0\ket{0}+b_1\ket{1} \big)c_0\ket{0}+
	\big( b_0\ket{0}+b_1\ket{1} \big)c_1\ket{1}\\
&=\big( b_0\ket{0}+b_1\ket{1} \big)\big( c_0\ket{0}+c_1\ket{1}\big)
=\ket{\phi}\otimes\ket{\psi}.
\end{align}
\end{subequations}
The form \eqref{eq:SWAP} of $SWAP$ gate was used.

\subsection{Exercise 5 (SWAP test)}\noindent
Writing in matrix form the circuit in fig. 1(d) and operating to get 
the state $\ket{\Psi'}$ just before the measurement,
\begin{subequations}
\begin{align}
\ket{\Psi'}&=\big( H\otimes\1_4 \big)	
	\big( \dyad{0}{0}\otimes\1_4+\dyad{1}{1}\otimes SWAP \big)
		\big(H\otimes\1_4 \big)\big[ \ket{0}\otimes\ket{\psi}\otimes\ket{\phi} \big]\\
&=\big( H\otimes\1_4 \big)	
	\big( \dyad{0}{0}\otimes\1_4+\dyad{1}{1}\otimes SWAP \big)
		\frac{1}{\sqrt{2}}\bigg[ \big(\ket{0}+\ket{1} \big)\ket{\psi}\ket{\phi} \bigg]\\
&=\big( H\otimes\1_4 \big)	
	\frac{1}{\sqrt{2}}\bigg[ \ket{0}\dket{\psi}{\phi}+\ket{1}\dket{\phi}{\psi} \bigg]\\
\ket{\Psi'}&=\frac{1}{2}\bigg[\big( \ket{0}+\ket{1} \big)\dket{\psi}{\phi} +
	\big( \ket{0}-\ket{1} \big)\dket{\phi}{\psi} \bigg].
\end{align}
\end{subequations}
Then, performing the measurement,
\begin{subequations}
\begin{align}
\expval{\sigma_z}&=\matrixel{\Psi'}{\sigma_z}{\Psi'}\\
&=\frac{1}{4}\Big( \braket{0\psi\phi}{0\psi\phi}+\braket{0\phi\psi}{0\psi\phi}
+\braket{0\psi\phi}{0\phi\psi}+\braket{0\phi\psi}{0\phi\psi}\\
&\qquad
 -\braket{1\psi\phi}{1\psi\phi}+\braket{1\phi\psi}{1\psi\phi}+
\braket{1\psi\phi}{1\phi\psi}-\braket{1\phi\psi}{1\phi\psi}
\Big)\\
&=\abs{\braket{\phi}{\psi}}^2.
\end{align}
\end{subequations}

\subsection{Exercise 6 (Useful equality).}\noindent
Let us start with the right-hand side of the equation and arrive at the left-hand side,
\begin{subequations}
\begin{align}
\label{eq:ex3.6:line1}
e^{i\pi \sigma_k/4}\rho e^{-i\pi \sigma_k/4}-
	e^{-i\pi \sigma_k/4}\rho e^{i\pi \sigma_k/4}&=
		\frac{1}{2}\big(\1+i\sigma_k \big)\rho\big(\1-i\sigma_k \big)-
			\frac{1}{2}\big(\1-i\sigma_k \big)\rho\big(\1+i\sigma_k \big)\\
&=\frac{1}{2}\big( -2i\rho\sigma_k+2i\sigma_k\rho \big)\\
&=i\comm{\sigma_k}{\rho},
\end{align}
\end{subequations}
where the relation $e^{\pm i\theta\sigma}=\cos \theta\1\pm i\sin\theta\sigma$
was used in \eqref{eq:ex3.6:line1}.

\subsection{Exercise 7 (Useful equality 2)}\noindent
Recall that $e^{\pm i\theta\sigma}=\cos \theta\1\pm i\sin\theta\sigma$, then 
\begin{align}\label{eq:ex7:left}
\pdv{e^{i\theta\sigma}}{\theta}=-\sin\theta\1-i\cos\theta\sigma.
\end{align}
On the other hand, 
\begin{align}\label{eq:ex7:right}
-i\sigma e^{-i\theta\sigma}=-i\cos\theta\sigma-\sin\theta\sigma^2.
\end{align}
Recall that $\sigma^2=\1$, then \eqref{eq:ex7:left} is equal to \eqref{eq:ex7:right}.

\subsection{Exercise 8 (Partial derivative of $U(\theta)$).}
Using the known relation $e^{\pm i\theta'\sigma}=\cos \theta'\1\pm i\sin\theta'\sigma$,
with $\theta'=\theta/2$
\begin{align}
\pdv{\theta}\bigg( \prod_{j=1}^Le^{-i\theta_j\sigma_{\mu}^j/2}W_j \bigg)&=
	e^{-i\theta_j\sigma_{\mu}^j/2}W_j\ldots 
		\pdv{\theta_k}\bigg( e^{-i\theta_k\sigma_{\mu}^k/2}W_k \bigg)
		\ldots e^{-i\theta_1\sigma_{\mu}^1/2}W_1,
\end{align}
using the previous result of exercise 7,
\begin{subequations}
\begin{align}
e^{-i\theta_j\sigma_{\mu}^j/2}W_j\ldots 
	\pdv{\theta_k}\bigg( e^{-i\theta_k\sigma_{\mu}^k/2}W_k \bigg)
		\ldots e^{-i\theta_1\sigma_{\mu}^1/2}W_1&=
			e^{-i\theta_j\sigma_{\mu}^j/2}W_j\ldots 
				\bigg( -\frac{i}{2}\sigma_{\mu}^ke^{-i\theta_k\sigma_{\mu}^k/2}W_k \bigg)
					\ldots e^{-i\theta_1\sigma_{\mu}^1/2}W_1\\
&=-\frac{i}{2}\prod_{j>k}e^{-i\theta_j\sigma_{\mu}^j/2}W_j\sigma_{\mu}^k
	\prod_{j\le k}e^{-i\theta_j\sigma_{\mu}^j/2}W_j.
\end{align}
\end{subequations}