\documentclass[10pt,a4paper]{article}
\usepackage[utf8]{inputenc}
\usepackage{authblk}
\usepackage[english]{babel}% \decimalpoint
\usepackage{amsmath}
\usepackage{amsfonts}
\usepackage{amssymb}
\usepackage{graphicx}
\usepackage{bbm}
\usepackage{float}
\usepackage{hyperref}
\hypersetup{
    colorlinks=true,
    linkcolor=black,
    filecolor=magenta,      
    urlcolor=blue,
    pdftitle={Overleaf Example},
    pdfpagemode=FullScreen,
    }
\usepackage{physics}
\usepackage[left=2cm,right=2cm,top=2cm,bottom=2cm]{geometry}

\usepackage{fancybox}
\usepackage{colortbl}
\usepackage{amsbsy}
\usepackage[draft,inline,nomargin]{fixme} \fxsetup{theme=color}
\FXRegisterAuthor{ja}{aja}{\color{black}\textbf{Respuestas}}

\newcommand{\uv}[1]{\vb{\hat{#1}}}
\newcommand{\points}[1]{\textcolor{red}{\text{(#1 puntos)}}}
\newcommand{\dket}[2]{\ket{#1}\ket{#2}}
\newcommand{\dbra}[2]{\bra{#1}\bra{#2}}
\newcommand{\1}{\mathbbm{1}}

\title{\textbf{Theoretical exercises}\\\vspace{2mm}
\large{PWF 2021 Quantum Machine Learning}}
\author[1]{José Alfredo de León\thanks{\href{mailto:deleongarrido.jose@gmail.com}{deleongarrido.jose@gmail.com}}}
\affil[1]{Escuela de Ciencias Físicas y Matemáticas - USAC, Guatemala}
\date{\today}
\begin{document}
\maketitle
\begin{abstract}
In this manuscript I present my solutions of the theoretical exercises of the 
PWF 2021 course ``Quantum Machine Learning'' imparted by Marco Cerezo.
Mathematica was used for several exercises and the Notebooks (one for each
of the exercise sheets) were uploaded to a public github repository 
(\href{https://github.com/deleonja/pwf2021_qml}{click here}) for anyone 
to access it and the rest of the material I used for the course.
\end{abstract}
\tableofcontents

\subsection*{Mathematica definitions}
Several exercises were solved in Mathematica. The following definitions 
where used troughout the Notebooks used (one for each exercise sheet).
I encourage you to go the github repository \href{https://github.com/deleonja/pwf2021_qml}
{github.com/deleonja/pwf2021\_qml} I created for anyone to find the Mathematica
notebooks, Tex files, and Jupyter notebook where the exercises of the course 
where solved by me.
\begin{verbatim}
(*Commutator*)
Comm[A_, B_] := A.B - B.A
(*Anti-commutator*)
AntiComm[A_, B_] := A.B + B.A

(*Pauli matrices and tensor products of them*)
Pauli[0] = Pauli[{0}] = {{1, 0}, {0, 1}};
Pauli[1] = Pauli[{1}] = {{0, 1}, {1, 0}};
Pauli[2] = Pauli[{2}] = {{0, -I}, {I, 0}};
Pauli[3] = Pauli[{3}] = {{1, 0}, {0, -1}};
Pauli[Indices_List] := KroneckerProduct @@ (Pauli /@ Indices);

(*Computational basis in Dirac notation*)
Ket[0] = {1, 0};
Ket[1] = {0, 1};

(*Rotation matrix in the Bloch sphere*)
U[i_, \[Theta]_] := 
 Cos[\[Theta]/2] IdentityMatrix[2] - I Sin[\[Theta]/2] Pauli[i]

(*General 1-qubit state in Dirac notation*)
Ket[\[Psi]] = \[Alpha] Ket[0] + \[Beta] Ket[1];
\end{verbatim}

\section{Exercise sheet 1}

\subsection{Exercise 1 (Properties of the Pauli Matrices)}\noindent 
Proved with brute force on Mathematica.
\subsubsection{a.}
\begin{verbatim}
Pauli[#].Pauli[#] == IdentityMatrix[2] & /@ Range[3]
Out:={True, True, True}
\end{verbatim}

\subsubsection{b.}
\begin{verbatim}
Tr[Pauli[#]] & /@ Range[3]
Out:={0, 0, 0}
\end{verbatim}

\subsubsection{c.}
\begin{verbatim}
(*Define Levi-civita symbol*)
\[Epsilon] = Normal[LeviCivitaTensor[3]];
Comm[Pauli[#[[1]]], Pauli[#[[2]]]] == 
   2 I \[Epsilon][[#[[1]], #[[2]], #[[3]]]] Pauli[#[[3]]] & /@ 
 Permutations[{1, 2, 3}]
Out:={True, True, True, True, True, True}
\end{verbatim}

\subsubsection{d.}
\begin{verbatim}
AntiComm[Pauli[#[[1]]], Pauli[#[[2]]]] == 
   2 I KroneckerDelta[#[[1]], #[[2]]] IdentityMatrix[2] & /@ 
 Permutations[{1, 2, 3}]
Out:={True, True, True, True, True, True}
\end{verbatim}

\subsection{Exercise 2 (Rotations on the Bloch Sphere)}\noindent
With Mathematica:
\begin{verbatim}
(*Denote Pauli[3] as Z (for convenience)*)
Z = Pauli[3];
Z.{1, 0} == {1, 0}
Z.{0, 1} == -{0, 1}
Out:=True
Out:=True
\end{verbatim}
Nevertheless, it is instructive to do this exercise using Dirac notation as follows. The Pauli matrix
$\sigma_z$ can be written as $\sigma_z=\dyad{0}{0}-\dyad{1}{1}$, then
\begin{align}
\sigma_z \ket{0}&=\qty(\dyad{0}{0}-\dyad{1}{1})\ket{0}=\ket{0}\\
\sigma_z \ket{1}&=\qty(\dyad{0}{0}-\dyad{1}{1})\ket{1}=-\ket{1},
\end{align}
where orthogonality relation $\braket{i}{j}=\delta_{ij}$ has ben employed.

\subsection{Exercise 3 (Density matrix)}\noindent
Since this exercise is one of Dirac notation practice, it was proved 
in Mathematica with the following code.
\begin{verbatim}
(*Define of ket psi in terms of azimuthal and polar angles*)
\[Psi] = Cos[\[Theta]/2] Ket[0] + E^(I \[Phi]) Sin[\[Theta]/2] Ket[1]
(*Outer product of psi*)
MatrixForm[Outer[Times, \[Psi], Conjugate[\[Psi]]]]
\end{verbatim}

\subsection{Exercise 4 (Rotations on the Bloch Sphere)}\noindent
Recall that, given an operator $A$, $f(A)=\sum_if(\lambda_i)\dyad{\phi_i}{\phi_i}$,
with $\lambda_i$ and $\ket{\phi_i}$ its eigenvalues and eigenvectors. Then, 
\begin{align}
e^{-i\theta\sigma/2}=\sum_{j}e^{-i\theta\lambda_j/2}\dyad{\phi_j}{\phi_j},
\end{align}
where $\qty{\ket{\phi_j}}$ is the eigenbasis of $\sigma$. Now, using the 
Euler's formula $e^{\pm i\theta}=\cos \theta\pm i\sin\theta$, the fact that 
all $\sigma$'s eigenvalues are $\lambda_{\pm}=\pm 1$, and the parity 
of sine and cosine functions,
\begin{subequations}
\begin{align}
\sum_{j}e^{-i\theta\lambda_j/2}\dyad{\phi_j}{\phi_j}&=
\qty(\cos\qty(\frac{\theta}{2})-i\sin\qty(\frac{\theta}{2}))\dyad{\phi_+}{\phi_+}
+\qty(\cos\qty(\frac{\theta}{2})+i\sin\qty(\frac{\theta}{2}))\dyad{\phi_-}{\phi_-}\\
&=\cos\qty(\frac{\theta}{2})\qty(\dyad{\phi_+}{\phi_+}+\dyad{\phi_-}{\phi_-})
-i\sin\qty(\frac{\theta}{2})\qty(\dyad{\phi_+}{\phi_+}-\dyad{\phi_-}{\phi_-})\label{eq:ex4:almost}\\
&=\cos\qty(\frac{\theta}{2})I-i\sin\qty(\frac{\theta}{2})\sigma,
\end{align}
\end{subequations}
where we have used for the cosine and sine terms in \eqref{eq:ex4:almost}
that $I=\sum_j\dyad{\phi_j}{\phi_j}$ for an orthonormal
basis $\qty{\ket{\phi_j}}$, and the spectral theorem (recall that 
$\lambda_{\pm}=\pm1$), respectively.

\subsection{Exercise 5 (Rotations about the y-axis)}
\begin{verbatim}
(*Definition of ket 0*)
Ket[0] = {1, 0};
(*Definition of rotation operator U*)
U[i_, \[Theta]_] :=Cos[\[Theta]/2] IdentityMatrix[2] - I Sin[\[Theta]/2] Pauli[i]
U[2, \[Pi]/2].Ket[0] == 1/Sqrt[2] (Ket[0] + Ket[1])
Out:= True
\end{verbatim}

\begin{verbatim}
(*Definition of function to construct tensor products of Pauli matrices*)
Pauli[0] = Pauli[{0}] = {{1, 0}, {0, 1}};
Pauli[1] = Pauli[{1}] = {{0, 1}, {1, 0}};
Pauli[2] = Pauli[{2}] = {{0, -I}, {I, 0}};
Pauli[3] = Pauli[{3}] = {{1, 0}, {0, -1}};
Pauli[Indices_List] := KroneckerProduct @@ (Pauli /@ Indices);
(*Denote Z to Pauli[3]*)
Z = Pauli[3];
(*Check that rotation of Z by U_y[pi/2] equals to X*)
U[2, \[Pi]/2].Z.ConjugateTranspose[U[2, \[Pi]/2]] == Pauli[1]
\end{verbatim}

\subsection{Exercise 6 (Change of basis)}\noindent
\subsubsection{a.}
$U_y\qty(\frac{\pi}{2})$

\subsubsection{b.}
Note that the rotation is done counter-clockwise, $U_x\qty(-\frac{\pi}{2})$.

\subsection{Exercise 7 (Expectation value)}
\begin{verbatim}
(*Define ket Psi*)
Ket[\[Psi]] = \[Alpha] Ket[0] + \[Beta] Ket[1];
(*Construct Rho by making the outer product of Psi with itself*)
\[Rho] = Outer[Times, Ket[\[Psi]], Conjugate[Ket[\[Psi]]]]
(*Check that <Z>=Tr(Rho.Z)*)
Conjugate[Ket[\[Psi]]].Z.Ket[\[Psi]] == Tr[\[Rho].Z]
Out:=True
(*Check that <Z>=|alpha|^2-|beta|^2*)
Conjugate[Ket[\[Psi]]].Z.Ket[\[Psi]] ==  Abs[\[Alpha]]^2 - Abs[\[Beta]]^2
(*Check that Tr(Rho.Z)=|alpha|^2-|beta|^2*)
Tr[\[Rho].Z] == Abs[\[Alpha]]^2 - Abs[\[Beta]]^2
\end{verbatim}

\subsection{Exercise 8 (Expectation value and change of basis)}\noindent
$\expval{X}=\expval{U_y^{\dagger}\qty(\frac{\pi}{2})ZU_y\qty(\frac{\pi}{2})}$
One makes a rotation of the state $\ket{\psi}$ of $\pi/2$ about the $y$ axis and 
then measures in the computational basis.

\subsection{Exercise 9 (Expectation values)}\noindent
Let us expand the state $\ket{\psi}$ in the Hamiltonian eigenbasis $\qty{\ket{E_i}}$
as $\ket{\psi}=\sum_i\braket{E_i}{\psi}\ket{E_i}$, then
\begin{subequations}
\begin{align}
\Tr \qty(\dyad{\psi}{\psi}H)
&=\sum_{i,j}\braket{\phi_i}{\psi}\braket{\psi}{\phi_j}\Tr \qty(\dyad{E_i}{E_j}H)\\
&=\sum_{i,j}\braket{\phi_i}{\psi}\braket{\psi}{\phi_j}\sum_k 
\matrixel{E_k}{\qty(\dyad{E_i}{E_j}H)}{E_k}\\
&=\sum_{i,j,k}\braket{\phi_i}{\psi}\braket{\psi}{\phi_j} \delta_{k,i}\delta_{j.k}E_k\\
&=\sum_i\abs{\braket{\phi_i}{\psi}}^2 E_i,\label{eq:ex9}
\end{align}
\end{subequations}
where the squared braket is real, and the eigenenergies $E_i$ too. Then, it is straightforward 
that the sum \eqref{eq:ex9} is a real number.

\subsection{Exercise 10 (Change of basis)}\noindent
By means of a substitution $\ket{0}=\qty(c_{00}\ket{E_0}+c_{01}\ket{E_1})$
and $\ket{1}=\qty(c_{10}\ket{E_0}+c_{11}\ket{E_1})$,
\begin{subequations}
\begin{align}
\ket{\psi}&=a\qty(c_{00}\ket{E_0}+c_{01}\ket{E_1})+
	b\qty(c_{10}\ket{E_0}+c_{11}\ket{E_1})\\
\ket{\psi}&=\qty(a\,c_{00}+b\,c_{10})\ket{E_0}+
	\qty(a\,c_{01}+b\,c_{11})\ket{E_1}.\label{eq:ex10}
\end{align}
\end{subequations}

\subsection{Exercise 11 (Change of basis and expectation value)}\noindent
Writing down the density matrix of $\ket{\psi}$ in \eqref{eq:ex10}
\begin{align}
\dyad{\psi}{\psi}&=\qty(a\,c_{00}+b\,c_{10})\qty(a\,c_{00}+b\,c_{10})^*\dyad{E_0}{E_0}+\nonumber\\
	&\hspace{5mm}\qty(a\,c_{00}+b\,c_{10})\qty(a\,c_{01}+b\,c_{11})^*\dyad{E_0}{E_1}+\nonumber\\ 
		&\hspace{5mm}\qty(a\,c_{01}+b\,c_{11})\qty(a\,c_{00}+b\,c_{10})^*\dyad{E_1}{E_0}+\nonumber\\
			&\hspace{5mm}\qty(a\,c_{01}+b\,c_{11})\qty(a\,c_{01}+b\,c_{11})^*\dyad{E_1}{E_1}.
\end{align}
Then, it follows 
\begin{subequations}
\begin{align}
\Tr\qty[\dyad{\psi}{\psi}H]&=
	\sum_{i=0}^1\matrixel{E_i}{\qty(\dyad{\psi}{\psi}H)}{E_i}\\
\Tr\qty[\dyad{\psi}{\psi}H]&=\qty(a\,c_{00}+b\,c_{10})\qty(a\,c_{00}+b\,c_{10})^*E_0+
		\qty(a\,c_{01}+b\,c_{11})\qty(a\,c_{01}+b\,c_{11})^*E_1.
\end{align}
\end{subequations}

\subsection{Exercise 12 (Mixed States)}\noindent
The density matrix $\rho$ of a system in a pure state $\ket{\psi}$ writes
$\rho=\dyad{\psi}{\psi}$, then it follows
\begin{align}
\rho^2=\qty(\dyad{\psi}{\psi})\qty(\dyad{\psi}{\psi})=\dyad{\psi}{\psi}=\rho,
\end{align}
where the orthonormality condition $\braket{\psi}{\psi}=1$ has been used.


\subsection{Exercise 13 (Mixed States 1)}\noindent
To find the eigenvalues of density matrix
\begin{align}
\rho=\frac{1}{2}\qty(I+\vec r\cdot\vec\sigma)
\end{align}
one solves the equation $\det\qty(I-\lambda\rho)=0$ for $\lambda$ and finds
\begin{align}
\lambda_{\pm}=\frac{1}{2}\qty(1\pm\sqrt{r_x^2+r_y^2+r_z^2}).
\end{align}
Now, using the requirement $\Tr\qty(\rho^2)=1$ for $\rho$ to be pure, we find 
that an equivalent condition in terms of the Bloch vector is that $\vec r$ must 
be of unit lenght. When $\abs{\vec r}<1$, the system is an mixed state.

\subsection{Exercise 14 (Mixed States 2)}\noindent
Using the general form of a qubit density matrix
\begin{align}
\rho=\mqty(a&b\\b^*&c)
\end{align}
one finds the Bloch component, via $r_i=\Tr \qty(\rho\,\sigma_i)$,
\begin{subequations}
\begin{align}
r_x&=2\Re(b),\\
r_y&=2i\Re(b),\\
r_z&=a-c.
\end{align}
\end{subequations}
\section{Exercise sheet 2}
\subsection{Exercise 1 (Tensor Product of Pauli matrices)}\noindent
Done in Mathematica with the following code
\begin{verbatim}
MatrixForm /@ {KroneckerProduct[Pauli[3], Pauli[0]], 
  KroneckerProduct[Pauli[0], Pauli[3]], 
  KroneckerProduct[Pauli[1], Pauli[1]]}
\end{verbatim}
and the output is
\begin{align}
\qty{
\left(
\begin{array}{cccc}
 1 & 0 & 0 & 0 \\
 0 & 1 & 0 & 0 \\
 0 & 0 & -1 & 0 \\
 0 & 0 & 0 & -1 \\
\end{array}
\right),
\left(
\begin{array}{cccc}
 1 & 0 & 0 & 0 \\
 0 & -1 & 0 & 0 \\
 0 & 0 & 1 & 0 \\
 0 & 0 & 0 & -1 \\
\end{array}
\right),
\left(
\begin{array}{cccc}
 0 & 0 & 0 & 1 \\
 0 & 0 & 1 & 0 \\
 0 & 1 & 0 & 0 \\
 1 & 0 & 0 & 0 \\
\end{array}
\right)}.
\end{align}

\subsection{Exercise 2 (Transpose, conjugate and dagger of tensor
products)}
\begin{align}
\qty(V\otimes W)^*=\mqty(V_{00}W&\dots&V_{0N}W\\ 
\vdots&\ddots&\vdots\\ V_{M0}W&\dots&V_{MN}W)^*=
\mqty(V_{00}^*W^*&\dots&V_{0N}^*W^*\\ 
\vdots&\ddots&\vdots\\
V_{M0}^*W^*&\dots&V_{MN}^*W^*)=V^*\otimes W^*
\end{align}

\begin{subequations}
\begin{align}
\qty(V\otimes W)^t&=\mqty(V_{00}W&\dots&V_{0N}W\\ 
\vdots&\ddots&\vdots\\ V_{M0}W&\dots&V_{MN}W)^t=
\mqty(V_{00}W_{00}&\dots&V_{00}W_{0q}&\dots&V_{0n}W_{00}&\dots&V_{0n}W_{0q}\\ 
\vdots&\ddots&\vdots&\ddots&\vdots&\ddots&\vdots\\
V_{00}W_{p0}&\dots&V_{00}W_{pq}&\dots&V_{0n}W_{p0}&\dots&V_{0n}W_{pq}\\ 
\vdots&\ddots&\vdots&\ddots&\vdots&\ddots&\vdots\\
V_{m0}W_{00}&\dots&V_{m0}W_{0q}&\dots&V_{mn}W_{00}&\dots&V_{mn}W_{0q}\\ 
\vdots&\ddots&\vdots&\ddots&\vdots&\ddots&\vdots\\
V_{m0}W_{p0}&\dots&V_{m0}W_{pq}&\dots&V_{mn}W_{p0}&\dots&V_{mn}W_{pq})^t\\
&=\mqty(V_{00}W_{00}&\dots&V_{00}W_{p0}&\dots&V_{m0}W_{00}&\dots&V_{m0}W_{p0}\\ 
\vdots&\ddots&\vdots&\ddots&\vdots&\ddots&\vdots\\
V_{00}W_{0q}&\dots&V_{00}W_{pq}&\dots&V_{m0}W_{0q}&\dots&V_{m0}W_{pq}\\ 
\vdots&\ddots&\vdots&\ddots&\vdots&\ddots&\vdots\\
V_{0n}W_{00}&\dots&V_{0n}W_{p0}&\dots&V_{mn}W_{00}&\dots&V_{mn}W_{p0}\\ 
\vdots&\ddots&\vdots&\ddots&\vdots&\ddots&\vdots\\
V_{0n}W_{0q}&\dots&V_{0n}W_{pq}&\dots&V_{mn}W_{0q}&\dots&V_{mn}W_{pq})
=V^t\otimes W^t
\end{align}
\end{subequations}

De los dos resultados anteriores es evidente que 
\begin{align}
\qty(V\otimes W)^{\dagger}=\qty(\qty(V\otimes W)^{t})^*
=\qty(V^t\otimes W^{t})^*=\qty(V^t)^*\otimes \qty(W^t)^*
=V^{\dagger}\otimes W^{\dagger}.
\end{align}

\subsection{Exercise 3 (Tensor product of unitary matrices)}
\begin{align}
\qty(U_1\otimes U_2)\qty(U_1\otimes U_2)^{\dagger}=
U_1U_1^{\dagger}\otimes U_2¸U_2^{\dagger}=
U_1^{\dagger}U_1\otimes U_2^{\dagger}U_2=I_1\otimes I_2
\end{align}

\subsection{Exercise 4 (Tensor product of Hermitian matrices)}
As a consequence of exercise 2,
$\qty(H_1\otimes H_2)^t=H_1^{\dagger}\otimes H_2^{\dagger}.$
Now, if $H_i$ are each Hermitian, then $\qty(H_1\otimes H_2)^t
=H_1\otimes H_2$, and $H_1\otimes H_2$ is Hermitian.

\subsection{Exercise 5 (Vector representation of the basis)}
\begin{verbatim}
MatrixForm /@ (computationalBasis = {Ket[0], Ket[1]})
Flatten[KroneckerProduct[computationalBasis[[#[[1]]]], 
     computationalBasis[[#[[2]]]]]] & /@
  Tuples[Range[2], 2];
MatrixForm /@ %
\end{verbatim}
Output:
\begin{align}
\qty{\mqty(1\\0\\0\\0),\mqty(0\\1\\0\\0),\mqty(0\\0\\1\\0),\mqty(0\\0\\0\\1)}
\end{align}

\subsection{Exercise 6 ($n$-qubit systems)}\noindent
There are $2^n$ elements in the basis of the Hilbert space.

\subsection{Exercise 7 (Expectation value)}
\begin{align}
\expval{Z\otimes Z}=\abs{c_{00}}^2-\abs{c_{01}}^2-\abs{c_{10}}^2+\abs{c_{11}}^2
\end{align}

\subsection{Exercise 8 (Marginal Probability)}
\textcolor{red}{Duda}

\subsection{Exercise 9 (Expectation value 2)}
Trivial from result in exercise 7.

\subsection{Exercise 10 (Partial Trace)}
Let the system state be $\ket{\psi}=\qty(\ket{00}+\ket{11})/2$, the 
density matrix writes
\begin{align}
\rho=\dyad{\psi}{\psi}=\frac{1}{2}\qty(\dyad{00}{00}+\dyad{11}{00}+
\dyad{00}{11}+\dyad{11}{11}).
\end{align}
Now, tracing over systems $A$ one gets the reduced density matrix of system $A$
as 
\begin{align}
\rho^A=\sum_{i=0}^1\matrixel{i^{(A)}}{\rho}{i^{(A)}},
\end{align}
where $\ket{i^{(A)}}$ denote the basis vectors (computational basis) of system $A$.
Thus, following an anologous program for $\rho^B$ one gets
\begin{align}
\rho^A=\rho^B=\frac{1}{2}\qty(\dyad{0}{0}+\dyad{1}{1})=\frac{I}{2}.
\end{align}

\subsection{Exercise 11 (Local unitaries)}
From local unitaries of the form $U\otimes V$ acting on $\ket{00}$ one only 
gets separable states of the form $U\ket{0}\otimes V\ket{0}$, which, by definition,
is not an entagled state. 

\subsection{Exercise 12 (Reduced States of pure states)}
\begin{align}
\ket{\psi}=\ket{v}\otimes\ket{w}\\
\rho=\dyad{v\otimes w}{v\otimes w}
\end{align}
If one constructs via appropiate rotations of the computational basis 
a basis with $\ket{v\otimes w}$ one of its elements, then it is trivial that 
reduced density matrices are
\begin{align}
\rho^A&=\dyad{v}{v} & \rho^B&=\dyad{w}{w}.
\end{align}
Both reduced density matrices satisfy the condition $\rho^2=\rho$, thus 
they both are pure states.

\subsection{Exercise 13 (Maximally entangled state)}\noindent
From the maximally entangled state of a bipartite system you obtain 
the maximally mixed state for both reduced density matrices. This means that
you do not have any information at all of the reduced states, because there is 
equal probability to measure the state $\ket{0}$ or $\ket{1}$.
\section{Exercise sheet 3}
\subsection{Exercise 1 (Hadamard test)}
Mathematically, what is happening in the circuit is
\begin{subequations}
\begin{align}
\qty(H\otimes\mathbbm{1})\qty(\dyad{0}{0}\otimes\mathbbm{1}+\dyad{1}{1}\otimes U)\qty(H\otimes \mathbbm{1})
\Big[\ket{0}\otimes \ket{\psi}\Big]&=
\qty(H\otimes\mathbbm{1})\qty(\dyad{0}{0}\otimes\mathbbm{1}+\dyad{1}{1}\otimes U)
\Big[\frac{1}{\sqrt{2}}\qty(\ket{0}+\ket{1})\ket{\psi}\Big]\\
&=\qty(H\otimes\mathbbm{1})\qty[\frac{1}{\sqrt{2}}\qty(\ket{0}\ket{\psi}+\ket{1}U\ket{\psi})]\\
&=\frac{1}{\sqrt{2}}
\qty(\ket{0}\qty(\mathbbm{1}+U)\ket{\psi}+
	\ket{1}\qty(\mathbbm{1}-U)\ket{\psi})
\end{align}
\end{subequations}
To find the probability the qubit $0$ in $\ket{0}$ is
\begin{align}
\matrixel{\psi'}{\qty(\dyad{0}{0}\otimes\mathbbm{1})}{\psi'}=
	\frac{1}{4}\qty(2+\matrixel{\psi}{U}{\psi}+\matrixel{\psi}{U^{\dagger}}{\psi})
\end{align}
Now, noting that $\qty(\matrixel{\psi}{U}{\psi})^*=\matrixel{\psi}{U^{\dagger}}{\psi}$,
\begin{align}
p(0)=\matrixel{\psi'}{\qty(\dyad{0}{0}\otimes\mathbbm{1})}{\psi'}=
	\frac{1}{2}\bigg(1+\Re \Big[\matrixel{\psi}{U}{\psi}\Big]\bigg).
\end{align}

\subsection{Exercise 2 (Error analysis)}

\subsection{Exercise 3 (Useful equality)}\noindent
The right-hand of the circuit equation in fig. 1(b) writes
\begin{align}
CNOT(0,1)\cdot CNOT(1,0)\cdot CNOT(0,1),
\end{align}
where $CNOT(c,t)$ is notation for the $CNOT$ gate with control qubit $c$
and target qubit $t$. The action of the $CNOT$ gate over the computational 
basis is $\ket{c}\ket{t}\mapsto\ket{c}\ket{t\oplus c}$ (with $\oplus$ sum
mod 2). Furthermore,
\begin{align}
CNOT(0,1)&=
\left(
\begin{array}{cccc}
 1 & 0 & 0 & 0 \\
 0 & 1 & 0 & 0 \\
 0 & 0 & 0 & 1 \\
 0 & 0 & 1 & 0 \\
\end{array}
\right),&
CNOT(1,0&)=
\left(
\begin{array}{cccc}
 1 & 0 & 0 & 0 \\
 0 & 0 & 0 & 1 \\
 0 & 0 & 1 & 0 \\
 0 & 1 & 0 & 0 \\
\end{array}
\right).
\end{align}
This exercised was done in Mathematica with the following code
\begin{verbatim}
(*Define CNOT(0,1), CNOT(1,0) and Swap*)
MatrixForm[
 CNOT0 = KroneckerProduct[M0, IdentityMatrix[2]] + 
   KroneckerProduct[M1, Pauli[1]]]
MatrixForm[
 CNOT1 = KroneckerProduct[IdentityMatrix[2], M0] + 
   KroneckerProduct[Pauli[1], M1]]
MatrixForm[
 Swap = {{1, 0, 0, 0}, {0, 0, 1, 0}, {0, 1, 0, 0}, {0, 0, 0, 1}}]
(*Do matrix multiplication*)
MatrixForm[CNOT0.CNOT1.CNOT0]
CNOT0.CNOT1.CNOT0 == Swap
Out:= True
\end{verbatim}

\subsection{Exercise 4 (Useful equality)}
$SWAP$ gate may be written in Dirac notation as
\begin{align}\label{eq:SWAP}
SWAP=\dyad{00}{00}+\dyad{01}{10}+\dyad{10}{01}+\dyad{11}{11}.
\end{align}
It the follows,
\begin{subequations}
\begin{align}
SWAP\Big[ \ket{\psi	}\otimes \ket{\phi}\Big]&=
	SWAP\Big[ c_0b_0\ket{00}+c_0b_1\ket{01}+c_1b_0\ket{10}+c_1b_1\ket{11} \Big]\\
&=c_0b_0\ket{00}+c_0b_1\ket{10}+c_1b_0\ket{01}+c_1b_1\ket{11}\\
&=\big( b_0\ket{0}+b_1\ket{1} \big)c_0\ket{0}+
	\big( b_0\ket{0}+b_1\ket{1} \big)c_1\ket{1}\\
&=\big( b_0\ket{0}+b_1\ket{1} \big)\big( c_0\ket{0}+c_1\ket{1}\big)
=\ket{\phi}\otimes\ket{\psi}.
\end{align}
\end{subequations}
The form \eqref{eq:SWAP} of $SWAP$ gate was used.

\subsection{Exercise 5 (SWAP test)}\noindent
Writing in matrix form the circuit in fig. 1(d) and operating to get 
the state $\ket{\Psi'}$ just before the measurement,
\begin{subequations}
\begin{align}
\ket{\Psi'}&=\big( H\otimes\1_4 \big)	
	\big( \dyad{0}{0}\otimes\1_4+\dyad{1}{1}\otimes SWAP \big)
		\big(H\otimes\1_4 \big)\big[ \ket{0}\otimes\ket{\psi}\otimes\ket{\phi} \big]\\
&=\big( H\otimes\1_4 \big)	
	\big( \dyad{0}{0}\otimes\1_4+\dyad{1}{1}\otimes SWAP \big)
		\frac{1}{\sqrt{2}}\bigg[ \big(\ket{0}+\ket{1} \big)\ket{\psi}\ket{\phi} \bigg]\\
&=\big( H\otimes\1_4 \big)	
	\frac{1}{\sqrt{2}}\bigg[ \ket{0}\dket{\psi}{\phi}+\ket{1}\dket{\phi}{\psi} \bigg]\\
\ket{\Psi'}&=\frac{1}{2}\bigg[\big( \ket{0}+\ket{1} \big)\dket{\psi}{\phi} +
	\big( \ket{0}-\ket{1} \big)\dket{\phi}{\psi} \bigg].
\end{align}
\end{subequations}
Then, performing the measurement,
\begin{subequations}
\begin{align}
\expval{\sigma_z}&=\matrixel{\Psi'}{\sigma_z}{\Psi'}\\
&=\frac{1}{4}\Big( \braket{0\psi\phi}{0\psi\phi}+\braket{0\phi\psi}{0\psi\phi}
+\braket{0\psi\phi}{0\phi\psi}+\braket{0\phi\psi}{0\phi\psi}\\
&\qquad
 -\braket{1\psi\phi}{1\psi\phi}+\braket{1\phi\psi}{1\psi\phi}+
\braket{1\psi\phi}{1\phi\psi}-\braket{1\phi\psi}{1\phi\psi}
\Big)\\
&=\abs{\braket{\phi}{\psi}}^2.
\end{align}
\end{subequations}

\subsection{Exercise 6 (Useful equality).}\noindent
Let us start with the right-hand side of the equation and arrive at the left-hand side,
\begin{subequations}
\begin{align}
\label{eq:ex3.6:line1}
e^{i\pi \sigma_k/4}\rho e^{-i\pi \sigma_k/4}-
	e^{-i\pi \sigma_k/4}\rho e^{i\pi \sigma_k/4}&=
		\frac{1}{2}\big(\1+i\sigma_k \big)\rho\big(\1-i\sigma_k \big)-
			\frac{1}{2}\big(\1-i\sigma_k \big)\rho\big(\1+i\sigma_k \big)\\
&=\frac{1}{2}\big( -2i\rho\sigma_k+2i\sigma_k\rho \big)\\
&=i\comm{\sigma_k}{\rho},
\end{align}
\end{subequations}
where the relation $e^{\pm i\theta\sigma}=\cos \theta\1\pm i\sin\theta\sigma$
was used in \eqref{eq:ex3.6:line1}.

\subsection{Exercise 7 (Useful equality 2)}\noindent
Recall that $e^{\pm i\theta\sigma}=\cos \theta\1\pm i\sin\theta\sigma$, then 
\begin{align}\label{eq:ex7:left}
\pdv{e^{i\theta\sigma}}{\theta}=-\sin\theta\1-i\cos\theta\sigma.
\end{align}
On the other hand, 
\begin{align}\label{eq:ex7:right}
-i\sigma e^{-i\theta\sigma}=-i\cos\theta\sigma-\sin\theta\sigma^2.
\end{align}
Recall that $\sigma^2=\1$, then \eqref{eq:ex7:left} is equal to \eqref{eq:ex7:right}.

\subsection{Exercise 8 (Partial derivative of $U(\theta)$).}
Using the known relation $e^{\pm i\theta'\sigma}=\cos \theta'\1\pm i\sin\theta'\sigma$,
with $\theta'=\theta/2$
\begin{align}
\pdv{\theta}\bigg( \prod_{j=1}^Le^{-i\theta_j\sigma_{\mu}^j/2}W_j \bigg)&=
	e^{-i\theta_j\sigma_{\mu}^j/2}W_j\ldots 
		\pdv{\theta_k}\bigg( e^{-i\theta_k\sigma_{\mu}^k/2}W_k \bigg)
		\ldots e^{-i\theta_1\sigma_{\mu}^1/2}W_1,
\end{align}
using the previous result of exercise 7,
\begin{subequations}
\begin{align}
e^{-i\theta_j\sigma_{\mu}^j/2}W_j\ldots 
	\pdv{\theta_k}\bigg( e^{-i\theta_k\sigma_{\mu}^k/2}W_k \bigg)
		\ldots e^{-i\theta_1\sigma_{\mu}^1/2}W_1&=
			e^{-i\theta_j\sigma_{\mu}^j/2}W_j\ldots 
				\bigg( -\frac{i}{2}\sigma_{\mu}^ke^{-i\theta_k\sigma_{\mu}^k/2}W_k \bigg)
					\ldots e^{-i\theta_1\sigma_{\mu}^1/2}W_1\\
&=-\frac{i}{2}\prod_{j>k}e^{-i\theta_j\sigma_{\mu}^j/2}W_j\sigma_{\mu}^k
	\prod_{j\le k}e^{-i\theta_j\sigma_{\mu}^j/2}W_j.
\end{align}
\end{subequations}

\end{document}