\subsection*{Mathematica definitions}
Several exercises were solved in Mathematica. The following definitions 
where used troughout the Notebooks used (one for each exercise sheet).
I encourage you to go the github repository \href{https://github.com/deleonja/pwf2021_qml}
{github.com/deleonja/pwf2021\_qml} I created for anyone to find the Mathematica
notebooks, Tex files, and Jupyter notebook where the exercises of the course 
where solved by me.
\begin{verbatim}
(*Commutator*)
Comm[A_, B_] := A.B - B.A
(*Anti-commutator*)
AntiComm[A_, B_] := A.B + B.A

(*Pauli matrices and tensor products of them*)
Pauli[0] = Pauli[{0}] = {{1, 0}, {0, 1}};
Pauli[1] = Pauli[{1}] = {{0, 1}, {1, 0}};
Pauli[2] = Pauli[{2}] = {{0, -I}, {I, 0}};
Pauli[3] = Pauli[{3}] = {{1, 0}, {0, -1}};
Pauli[Indices_List] := KroneckerProduct @@ (Pauli /@ Indices);

(*Computational basis in Dirac notation*)
Ket[0] = {1, 0};
Ket[1] = {0, 1};

(*Rotation matrix in the Bloch sphere*)
U[i_, \[Theta]_] := 
 Cos[\[Theta]/2] IdentityMatrix[2] - I Sin[\[Theta]/2] Pauli[i]

(*General 1-qubit state in Dirac notation*)
Ket[\[Psi]] = \[Alpha] Ket[0] + \[Beta] Ket[1];
\end{verbatim}

\section{Exercise sheet 1}

\subsection{Exercise 1 (Properties of the Pauli Matrices)}\noindent 
Proved with brute force on Mathematica.
\subsubsection{a.}
\begin{verbatim}
Pauli[#].Pauli[#] == IdentityMatrix[2] & /@ Range[3]
Out:={True, True, True}
\end{verbatim}

\subsubsection{b.}
\begin{verbatim}
Tr[Pauli[#]] & /@ Range[3]
Out:={0, 0, 0}
\end{verbatim}

\subsubsection{c.}
\begin{verbatim}
(*Define Levi-civita symbol*)
\[Epsilon] = Normal[LeviCivitaTensor[3]];
Comm[Pauli[#[[1]]], Pauli[#[[2]]]] == 
   2 I \[Epsilon][[#[[1]], #[[2]], #[[3]]]] Pauli[#[[3]]] & /@ 
 Permutations[{1, 2, 3}]
Out:={True, True, True, True, True, True}
\end{verbatim}

\subsubsection{d.}
\begin{verbatim}
AntiComm[Pauli[#[[1]]], Pauli[#[[2]]]] == 
   2 I KroneckerDelta[#[[1]], #[[2]]] IdentityMatrix[2] & /@ 
 Permutations[{1, 2, 3}]
Out:={True, True, True, True, True, True}
\end{verbatim}

\subsection{Exercise 2 (Rotations on the Bloch Sphere)}\noindent
With Mathematica:
\begin{verbatim}
(*Denote Pauli[3] as Z (for convenience)*)
Z = Pauli[3];
Z.{1, 0} == {1, 0}
Z.{0, 1} == -{0, 1}
Out:=True
Out:=True
\end{verbatim}
Nevertheless, it is instructive to do this exercise using Dirac notation as follows. The Pauli matrix
$\sigma_z$ can be written as $\sigma_z=\dyad{0}{0}-\dyad{1}{1}$, then
\begin{align}
\sigma_z \ket{0}&=\qty(\dyad{0}{0}-\dyad{1}{1})\ket{0}=\ket{0}\\
\sigma_z \ket{1}&=\qty(\dyad{0}{0}-\dyad{1}{1})\ket{1}=-\ket{1},
\end{align}
where orthogonality relation $\braket{i}{j}=\delta_{ij}$ has ben employed.

\subsection{Exercise 3 (Density matrix)}\noindent
Since this exercise is one of Dirac notation practice, it was proved 
in Mathematica with the following code.
\begin{verbatim}
(*Define of ket psi in terms of azimuthal and polar angles*)
\[Psi] = Cos[\[Theta]/2] Ket[0] + E^(I \[Phi]) Sin[\[Theta]/2] Ket[1]
(*Outer product of psi*)
MatrixForm[Outer[Times, \[Psi], Conjugate[\[Psi]]]]
\end{verbatim}

\subsection{Exercise 4 (Rotations on the Bloch Sphere)}\noindent
Recall that, given an operator $A$, $f(A)=\sum_if(\lambda_i)\dyad{\phi_i}{\phi_i}$,
with $\lambda_i$ and $\ket{\phi_i}$ its eigenvalues and eigenvectors. Then, 
\begin{align}
e^{-i\theta\sigma/2}=\sum_{j}e^{-i\theta\lambda_j/2}\dyad{\phi_j}{\phi_j},
\end{align}
where $\qty{\ket{\phi_j}}$ is the eigenbasis of $\sigma$. Now, using the 
Euler's formula $e^{\pm i\theta}=\cos \theta\pm i\sin\theta$, the fact that 
all $\sigma$'s eigenvalues are $\lambda_{\pm}=\pm 1$, and the parity 
of sine and cosine functions,
\begin{subequations}
\begin{align}
\sum_{j}e^{-i\theta\lambda_j/2}\dyad{\phi_j}{\phi_j}&=
\qty(\cos\qty(\frac{\theta}{2})-i\sin\qty(\frac{\theta}{2}))\dyad{\phi_+}{\phi_+}
+\qty(\cos\qty(\frac{\theta}{2})+i\sin\qty(\frac{\theta}{2}))\dyad{\phi_-}{\phi_-}\\
&=\cos\qty(\frac{\theta}{2})\qty(\dyad{\phi_+}{\phi_+}+\dyad{\phi_-}{\phi_-})
-i\sin\qty(\frac{\theta}{2})\qty(\dyad{\phi_+}{\phi_+}-\dyad{\phi_-}{\phi_-})\label{eq:ex4:almost}\\
&=\cos\qty(\frac{\theta}{2})I-i\sin\qty(\frac{\theta}{2})\sigma,
\end{align}
\end{subequations}
where we have used for the cosine and sine terms in \eqref{eq:ex4:almost}
that $I=\sum_j\dyad{\phi_j}{\phi_j}$ for an orthonormal
basis $\qty{\ket{\phi_j}}$, and the spectral theorem (recall that 
$\lambda_{\pm}=\pm1$), respectively.

\subsection{Exercise 5 (Rotations about the y-axis)}
\begin{verbatim}
(*Definition of ket 0*)
Ket[0] = {1, 0};
(*Definition of rotation operator U*)
U[i_, \[Theta]_] :=Cos[\[Theta]/2] IdentityMatrix[2] - I Sin[\[Theta]/2] Pauli[i]
U[2, \[Pi]/2].Ket[0] == 1/Sqrt[2] (Ket[0] + Ket[1])
Out:= True
\end{verbatim}

\begin{verbatim}
(*Definition of function to construct tensor products of Pauli matrices*)
Pauli[0] = Pauli[{0}] = {{1, 0}, {0, 1}};
Pauli[1] = Pauli[{1}] = {{0, 1}, {1, 0}};
Pauli[2] = Pauli[{2}] = {{0, -I}, {I, 0}};
Pauli[3] = Pauli[{3}] = {{1, 0}, {0, -1}};
Pauli[Indices_List] := KroneckerProduct @@ (Pauli /@ Indices);
(*Denote Z to Pauli[3]*)
Z = Pauli[3];
(*Check that rotation of Z by U_y[pi/2] equals to X*)
U[2, \[Pi]/2].Z.ConjugateTranspose[U[2, \[Pi]/2]] == Pauli[1]
\end{verbatim}

\subsection{Exercise 6 (Change of basis)}\noindent
\subsubsection{a.}
$U_y\qty(\frac{\pi}{2})$

\subsubsection{b.}
Note that the rotation is done counter-clockwise, $U_x\qty(-\frac{\pi}{2})$.

\subsection{Exercise 7 (Expectation value)}
\begin{verbatim}
(*Define ket Psi*)
Ket[\[Psi]] = \[Alpha] Ket[0] + \[Beta] Ket[1];
(*Construct Rho by making the outer product of Psi with itself*)
\[Rho] = Outer[Times, Ket[\[Psi]], Conjugate[Ket[\[Psi]]]]
(*Check that <Z>=Tr(Rho.Z)*)
Conjugate[Ket[\[Psi]]].Z.Ket[\[Psi]] == Tr[\[Rho].Z]
Out:=True
(*Check that <Z>=|alpha|^2-|beta|^2*)
Conjugate[Ket[\[Psi]]].Z.Ket[\[Psi]] ==  Abs[\[Alpha]]^2 - Abs[\[Beta]]^2
(*Check that Tr(Rho.Z)=|alpha|^2-|beta|^2*)
Tr[\[Rho].Z] == Abs[\[Alpha]]^2 - Abs[\[Beta]]^2
\end{verbatim}

\subsection{Exercise 8 (Expectation value and change of basis)}\noindent
$\expval{X}=\expval{U_y^{\dagger}\qty(\frac{\pi}{2})ZU_y\qty(\frac{\pi}{2})}$
One makes a rotation of the state $\ket{\psi}$ of $\pi/2$ about the $y$ axis and 
then measures in the computational basis.

\subsection{Exercise 9 (Expectation values)}\noindent
Let us expand the state $\ket{\psi}$ in the Hamiltonian eigenbasis $\qty{\ket{E_i}}$
as $\ket{\psi}=\sum_i\braket{E_i}{\psi}\ket{E_i}$, then
\begin{subequations}
\begin{align}
\Tr \qty(\dyad{\psi}{\psi}H)
&=\sum_{i,j}\braket{\phi_i}{\psi}\braket{\psi}{\phi_j}\Tr \qty(\dyad{E_i}{E_j}H)\\
&=\sum_{i,j}\braket{\phi_i}{\psi}\braket{\psi}{\phi_j}\sum_k 
\matrixel{E_k}{\qty(\dyad{E_i}{E_j}H)}{E_k}\\
&=\sum_{i,j,k}\braket{\phi_i}{\psi}\braket{\psi}{\phi_j} \delta_{k,i}\delta_{j.k}E_k\\
&=\sum_i\abs{\braket{\phi_i}{\psi}}^2 E_i,\label{eq:ex9}
\end{align}
\end{subequations}
where the squared braket is real, and the eigenenergies $E_i$ too. Then, it is straightforward 
that the sum \eqref{eq:ex9} is a real number.

\subsection{Exercise 10 (Change of basis)}\noindent
By means of a substitution $\ket{0}=\qty(c_{00}\ket{E_0}+c_{01}\ket{E_1})$
and $\ket{1}=\qty(c_{10}\ket{E_0}+c_{11}\ket{E_1})$,
\begin{subequations}
\begin{align}
\ket{\psi}&=a\qty(c_{00}\ket{E_0}+c_{01}\ket{E_1})+
	b\qty(c_{10}\ket{E_0}+c_{11}\ket{E_1})\\
\ket{\psi}&=\qty(a\,c_{00}+b\,c_{10})\ket{E_0}+
	\qty(a\,c_{01}+b\,c_{11})\ket{E_1}.\label{eq:ex10}
\end{align}
\end{subequations}

\subsection{Exercise 11 (Change of basis and expectation value)}\noindent
Writing down the density matrix of $\ket{\psi}$ in \eqref{eq:ex10}
\begin{align}
\dyad{\psi}{\psi}&=\qty(a\,c_{00}+b\,c_{10})\qty(a\,c_{00}+b\,c_{10})^*\dyad{E_0}{E_0}+\nonumber\\
	&\hspace{5mm}\qty(a\,c_{00}+b\,c_{10})\qty(a\,c_{01}+b\,c_{11})^*\dyad{E_0}{E_1}+\nonumber\\ 
		&\hspace{5mm}\qty(a\,c_{01}+b\,c_{11})\qty(a\,c_{00}+b\,c_{10})^*\dyad{E_1}{E_0}+\nonumber\\
			&\hspace{5mm}\qty(a\,c_{01}+b\,c_{11})\qty(a\,c_{01}+b\,c_{11})^*\dyad{E_1}{E_1}.
\end{align}
Then, it follows 
\begin{subequations}
\begin{align}
\Tr\qty[\dyad{\psi}{\psi}H]&=
	\sum_{i=0}^1\matrixel{E_i}{\qty(\dyad{\psi}{\psi}H)}{E_i}\\
\Tr\qty[\dyad{\psi}{\psi}H]&=\qty(a\,c_{00}+b\,c_{10})\qty(a\,c_{00}+b\,c_{10})^*E_0+
		\qty(a\,c_{01}+b\,c_{11})\qty(a\,c_{01}+b\,c_{11})^*E_1.
\end{align}
\end{subequations}

\subsection{Exercise 12 (Mixed States)}\noindent
The density matrix $\rho$ of a system in a pure state $\ket{\psi}$ writes
$\rho=\dyad{\psi}{\psi}$, then it follows
\begin{align}
\rho^2=\qty(\dyad{\psi}{\psi})\qty(\dyad{\psi}{\psi})=\dyad{\psi}{\psi}=\rho,
\end{align}
where the orthonormality condition $\braket{\psi}{\psi}=1$ has been used.


\subsection{Exercise 13 (Mixed States 1)}\noindent
To find the eigenvalues of density matrix
\begin{align}
\rho=\frac{1}{2}\qty(I+\vec r\cdot\vec\sigma)
\end{align}
one solves the equation $\det\qty(I-\lambda\rho)=0$ for $\lambda$ and finds
\begin{align}
\lambda_{\pm}=\frac{1}{2}\qty(1\pm\sqrt{r_x^2+r_y^2+r_z^2}).
\end{align}
Now, using the requirement $\Tr\qty(\rho^2)=1$ for $\rho$ to be pure, we find 
that an equivalent condition in terms of the Bloch vector is that $\vec r$ must 
be of unit lenght. When $\abs{\vec r}<1$, the system is an mixed state.

\subsection{Exercise 14 (Mixed States 2)}\noindent
Using the general form of a qubit density matrix
\begin{align}
\rho=\mqty(a&b\\b^*&c)
\end{align}
one finds the Bloch component, via $r_i=\Tr \qty(\rho\,\sigma_i)$,
\begin{subequations}
\begin{align}
r_x&=2\Re(b),\\
r_y&=2i\Re(b),\\
r_z&=a-c.
\end{align}
\end{subequations}