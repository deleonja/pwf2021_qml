\section{Exercise sheet 2}
\subsection{Exercise 1 (Tensor Product of Pauli matrices)}\noindent
Done in Mathematica with the following code
\begin{verbatim}
MatrixForm /@ {KroneckerProduct[Pauli[3], Pauli[0]], 
  KroneckerProduct[Pauli[0], Pauli[3]], 
  KroneckerProduct[Pauli[1], Pauli[1]]}
\end{verbatim}
and the output is
\begin{align}
\qty{
\left(
\begin{array}{cccc}
 1 & 0 & 0 & 0 \\
 0 & 1 & 0 & 0 \\
 0 & 0 & -1 & 0 \\
 0 & 0 & 0 & -1 \\
\end{array}
\right),
\left(
\begin{array}{cccc}
 1 & 0 & 0 & 0 \\
 0 & -1 & 0 & 0 \\
 0 & 0 & 1 & 0 \\
 0 & 0 & 0 & -1 \\
\end{array}
\right),
\left(
\begin{array}{cccc}
 0 & 0 & 0 & 1 \\
 0 & 0 & 1 & 0 \\
 0 & 1 & 0 & 0 \\
 1 & 0 & 0 & 0 \\
\end{array}
\right)}.
\end{align}

\subsection{Exercise 2 (Transpose, conjugate and dagger of tensor
products)}
\begin{align}
\qty(V\otimes W)^*=\mqty(V_{00}W&\dots&V_{0N}W\\ 
\vdots&\ddots&\vdots\\ V_{M0}W&\dots&V_{MN}W)^*=
\mqty(V_{00}^*W^*&\dots&V_{0N}^*W^*\\ 
\vdots&\ddots&\vdots\\
V_{M0}^*W^*&\dots&V_{MN}^*W^*)=V^*\otimes W^*
\end{align}

\begin{subequations}
\begin{align}
\qty(V\otimes W)^t&=\mqty(V_{00}W&\dots&V_{0N}W\\ 
\vdots&\ddots&\vdots\\ V_{M0}W&\dots&V_{MN}W)^t=
\mqty(V_{00}W_{00}&\dots&V_{00}W_{0q}&\dots&V_{0n}W_{00}&\dots&V_{0n}W_{0q}\\ 
\vdots&\ddots&\vdots&\ddots&\vdots&\ddots&\vdots\\
V_{00}W_{p0}&\dots&V_{00}W_{pq}&\dots&V_{0n}W_{p0}&\dots&V_{0n}W_{pq}\\ 
\vdots&\ddots&\vdots&\ddots&\vdots&\ddots&\vdots\\
V_{m0}W_{00}&\dots&V_{m0}W_{0q}&\dots&V_{mn}W_{00}&\dots&V_{mn}W_{0q}\\ 
\vdots&\ddots&\vdots&\ddots&\vdots&\ddots&\vdots\\
V_{m0}W_{p0}&\dots&V_{m0}W_{pq}&\dots&V_{mn}W_{p0}&\dots&V_{mn}W_{pq})^t\\
&=\mqty(V_{00}W_{00}&\dots&V_{00}W_{p0}&\dots&V_{m0}W_{00}&\dots&V_{m0}W_{p0}\\ 
\vdots&\ddots&\vdots&\ddots&\vdots&\ddots&\vdots\\
V_{00}W_{0q}&\dots&V_{00}W_{pq}&\dots&V_{m0}W_{0q}&\dots&V_{m0}W_{pq}\\ 
\vdots&\ddots&\vdots&\ddots&\vdots&\ddots&\vdots\\
V_{0n}W_{00}&\dots&V_{0n}W_{p0}&\dots&V_{mn}W_{00}&\dots&V_{mn}W_{p0}\\ 
\vdots&\ddots&\vdots&\ddots&\vdots&\ddots&\vdots\\
V_{0n}W_{0q}&\dots&V_{0n}W_{pq}&\dots&V_{mn}W_{0q}&\dots&V_{mn}W_{pq})
=V^t\otimes W^t
\end{align}
\end{subequations}

De los dos resultados anteriores es evidente que 
\begin{align}
\qty(V\otimes W)^{\dagger}=\qty(\qty(V\otimes W)^{t})^*
=\qty(V^t\otimes W^{t})^*=\qty(V^t)^*\otimes \qty(W^t)^*
=V^{\dagger}\otimes W^{\dagger}.
\end{align}

\subsection{Exercise 3 (Tensor product of unitary matrices)}
\begin{align}
\qty(U_1\otimes U_2)\qty(U_1\otimes U_2)^{\dagger}=
U_1U_1^{\dagger}\otimes U_2¸U_2^{\dagger}=
U_1^{\dagger}U_1\otimes U_2^{\dagger}U_2=I_1\otimes I_2
\end{align}

\subsection{Exercise 4 (Tensor product of Hermitian matrices)}
As a consequence of exercise 2,
$\qty(H_1\otimes H_2)^t=H_1^{\dagger}\otimes H_2^{\dagger}.$
Now, if $H_i$ are each Hermitian, then $\qty(H_1\otimes H_2)^t
=H_1\otimes H_2$, and $H_1\otimes H_2$ is Hermitian.

\subsection{Exercise 5 (Vector representation of the basis)}
\begin{verbatim}
MatrixForm /@ (computationalBasis = {Ket[0], Ket[1]})
Flatten[KroneckerProduct[computationalBasis[[#[[1]]]], 
     computationalBasis[[#[[2]]]]]] & /@
  Tuples[Range[2], 2];
MatrixForm /@ %
\end{verbatim}
Output:
\begin{align}
\qty{\mqty(1\\0\\0\\0),\mqty(0\\1\\0\\0),\mqty(0\\0\\1\\0),\mqty(0\\0\\0\\1)}
\end{align}

\subsection{Exercise 6 ($n$-qubit systems)}\noindent
There are $2^n$ elements in the basis of the Hilbert space.

\subsection{Exercise 7 (Expectation value)}
\begin{align}
\expval{Z\otimes Z}=\abs{c_{00}}^2-\abs{c_{01}}^2-\abs{c_{10}}^2+\abs{c_{11}}^2
\end{align}

\subsection{Exercise 8 (Marginal Probability)}
\textcolor{red}{Duda}

\subsection{Exercise 9 (Expectation value 2)}
Trivial from result in exercise 7.

\subsection{Exercise 10 (Partial Trace)}
Let the system state be $\ket{\psi}=\qty(\ket{00}+\ket{11})/2$, the 
density matrix writes
\begin{align}
\rho=\dyad{\psi}{\psi}=\frac{1}{2}\qty(\dyad{00}{00}+\dyad{11}{00}+
\dyad{00}{11}+\dyad{11}{11}).
\end{align}
Now, tracing over systems $A$ one gets the reduced density matrix of system $A$
as 
\begin{align}
\rho^A=\sum_{i=0}^1\matrixel{i^{(A)}}{\rho}{i^{(A)}},
\end{align}
where $\ket{i^{(A)}}$ denote the basis vectors (computational basis) of system $A$.
Thus, following an anologous program for $\rho^B$ one gets
\begin{align}
\rho^A=\rho^B=\frac{1}{2}\qty(\dyad{0}{0}+\dyad{1}{1})=\frac{I}{2}.
\end{align}

\subsection{Exercise 11 (Local unitaries)}
From local unitaries of the form $U\otimes V$ acting on $\ket{00}$ one only 
gets separable states of the form $U\ket{0}\otimes V\ket{0}$, which, by definition,
is not an entagled state. 

\subsection{Exercise 12 (Reduced States of pure states)}
\begin{align}
\ket{\psi}=\ket{v}\otimes\ket{w}\\
\rho=\dyad{v\otimes w}{v\otimes w}
\end{align}
If one constructs via appropiate rotations of the computational basis 
a basis with $\ket{v\otimes w}$ one of its elements, then it is trivial that 
reduced density matrices are
\begin{align}
\rho^A&=\dyad{v}{v} & \rho^B&=\dyad{w}{w}.
\end{align}
Both reduced density matrices satisfy the condition $\rho^2=\rho$, thus 
they both are pure states.

\subsection{Exercise 13 (Maximally entangled state)}\noindent
From the maximally entangled state of a bipartite system you obtain 
the maximally mixed state for both reduced density matrices. This means that
you do not have any information at all of the reduced states, because there is 
equal probability to measure the state $\ket{0}$ or $\ket{1}$.